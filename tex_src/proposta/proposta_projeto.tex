\documentclass[12pt,a4paper]{article}
\usepackage[latin1]{inputenc}


\title{Proposta de projeto para a disciplina EEL7801  \\
	   Projeto em Eletr�nica I\\ \vfill
	   \normalsize{Universidade Federal de Santa Catarina - UFSC \\
	   			   Professora: Daniela Ota Hisayasu Suzuki}
   	  }
     
\date{4 de Abril de 2019}
\begin{document}
	\maketitle 
	\paragraph{Proposta:}
	Nosso projeto prop�e um canal comunica��o digital via som, usando modula��o e demodula��o de sinais. O objetivo �, no final do projeto, obter uma comunica��o digital unidirecional est�vel e segura de m�dio alcance entre dois dispositivos, com toda a computa��o sendo contida nos pr�prios dispostivos para uso standalone.
	\section{Integrantes do Grupo}
	\begin{itemize}
		\item{Luiz Augusto Frazatto Fernandes: \it{17202752}}
		\item{Leonardo Jos� Held: \it{17203984}}
	\end{itemize}

	\section{Metodologia}
	O projeto passar� incialmente por uma fase de simula��o, onde ser�o desenvolvidos os algoritmos necess�rios para as fases de \{de\}modula��o. Ap�s a valida��o ser conclu�da e o algoritmo escolhido, passaremos a escolher e programar dois microcontroladores que escalam bem para aplica��es em tempo real. Um deles ser� o modulador, que usar� um transdutor para transmitir as ondas de som e o outro ser� um demodulador, que far� a parte de recep��o e demodula��o das ondas em bits novamente.
	
	
\end{document}