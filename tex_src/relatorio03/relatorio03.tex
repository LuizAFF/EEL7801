\documentclass[11pt,a4paper]{report}
\usepackage[utf8]{inputenc}
\usepackage{amsmath}
\usepackage{amsfonts}
\usepackage{amssymb}
\usepackage{graphicx}
\usepackage{pgfplots}
\usepackage{circuitikz}
\usepackage{hyperref}
\usepackage{tikz}
\usepackage{pdfpages}
\usepackage{minted}
\usepackage{mathtools}
\definecolor{bg}{rgb}{0.95,0.95,0.95}
\usetikzlibrary{calc,trees,positioning,arrows,chains,shapes.geometric,%
	decorations.pathreplacing,decorations.pathmorphing,shapes,%
	matrix,shapes.symbols}

\title{Relatório Final  \\
	Projeto em Eletrônica I - EEL7801 \\ \vfill
	\normalsize{Universidade Federal de Santa Catarina - UFSC \\
		Professora: Daniela Ota Hisayasu Suzuki}
	\author{
		{Luiz Augusto Frazatto Fernandes: \it{17202752}} \\
		{Leonardo José Held: \it{17203984}}
}
}
\date{6 de Julho de 2019}
\begin{document}
	\maketitle
	\setcounter{chapter}{0}

	
	\chapter{Implementação Algorítimica}
		A implementação algorítmica teve, como principais desafios, a tradução de códigos de alto nível de modulação e demodulação em códigos de baixo nível, bem como a adequação dos códigos aos MCUs escolhidos. A demodulação, no entanto, não fora completamente finalizada por conta de dois fatores:
		\begin{list}
			\item[1.] Uma das placas tornou-se irresponsiva e não mais conseguimos programá-la.
			\item[2.] Há a necessidade de grande capacidade de memória RAM para se processar o vetor a ser demodulado. Logo, fizemos o uso de um Raspberry Pi 3.
		\end{list}
		
	\section{Modulação em baixo nível (C/C++)}
	\paragraph{}
		A modulação é realizada através de duas Look Up Tables (LUT) contendo valores de duas funções seno (cada um com uma frequência diferente). Resumidamente: a informação (convertida em binário) é transformada num sinal modulado por FSK.
	\subsection{lut.h}
		\begin{minted}[frame=lines,
		framesep=2mm,
		baselinestretch=1.2,
		bgcolor=bg,
		fontsize=\footnotesize,
		linenos
		]{Cpp}
		
		
		\end{minted}
		
	\subsection{lut.c}
		\begin{minted}[frame=lines,
		framesep=2mm,
		baselinestretch=1.2,
		bgcolor=bg,
		fontsize=\footnotesize,
		linenos
		]{Cpp}
		
		
		\end{minted}
		
	\subsection{mod_main.c}
		\begin{minted}[frame=lines,
		framesep=2mm,
		baselinestretch=1.2,
		bgcolor=bg,
		fontsize=\footnotesize,
		linenos
		]{Cpp}
		
		
		\end{minted}
		
	\section{Aquisição dos dados}


	\section{Demodulação em baixo nível (C/C++)}
	\paragraph{}
		A demodulação consiste em, a partir dos dados obtidos pelo sensor, interpretar o sinal recebido e identificar a frequência com que esse "cruza" a faixa de DC OFFSET. Calcula-se o valor médio da tensão gerada pelo sinal e, a partir desse, consegue-se identificar diferentes frequências.
	
	\subsection{functions.h}
		\begin{minted}[frame=lines,
		framesep=2mm,
		baselinestretch=1.2,
		bgcolor=bg,
		fontsize=\footnotesize,
		linenos
		]{Cpp}
		
		
		\end{minted}
		
	
	\subsection{functions.c}
		\begin{minted}[frame=lines,
		framesep=2mm,
		baselinestretch=1.2,
		bgcolor=bg,
		fontsize=\footnotesize,
		linenos
		]{Cpp}
		
		
		\end{minted}
		
		
	\subsection{demod_main.c}
		\begin{minted}[frame=lines,
		framesep=2mm,
		baselinestretch=1.2,
		bgcolor=bg,
		fontsize=\footnotesize,
		linenos
		]{Cpp}
		
		
		\end{minted}
		
	

	
		
\end{document}