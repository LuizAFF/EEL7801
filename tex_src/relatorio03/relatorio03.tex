\documentclass[11pt,a4paper]{report}
\usepackage[utf8]{inputenc}
\usepackage{amsmath}
\usepackage{amsfonts}
\usepackage{amssymb}
\usepackage{graphicx}
\usepackage{pgfplots}
\usepackage{circuitikz}
\usepackage{hyperref}
\usepackage{tikz}
\usepackage{pdfpages}
\usepackage{minted}
\usepackage{mathtools}
\definecolor{bg}{rgb}{0.95,0.95,0.95}
\usetikzlibrary{calc,trees,positioning,arrows,chains,shapes.geometric,%
	decorations.pathreplacing,decorations.pathmorphing,shapes,%
	matrix,shapes.symbols}

\title{Relatório Final  \\
	Projeto em Eletrônica I - EEL7801 \\ \vfill
	\normalsize{Universidade Federal de Santa Catarina - UFSC \\
		Professora: Daniela Ota Hisayasu Suzuki}
	\author{
		{Luiz Augusto Frazatto Fernandes: \it{17202752}} \\
		{Leonardo José Held: \it{17203984}}
}
}
\date{6 de Julho de 2019}
\begin{document}
	\maketitle
	\setcounter{chapter}{0}

	
	\chapter{Implementação Algorítimica}
		A implementação algorítmica teve, como principais desafios, a tradução de códigos de alto nível de modulação e demodulação em códigos de baixo nível, bem como a adequação dos códigos aos MCUs escolhidos. A demodulação, no entanto, não fora completamente finalizada por conta de dois fatores:
		\begin{itemize}
			\item[1.] Uma das placas tornou-se irresponsiva e não mais conseguimos programá-la.
			\item[2.] Há a necessidade de grande capacidade de memória RAM para se processar o vetor a ser demodulado. Logo, fizemos o uso de um Raspberry Pi 3.
		\end{itemize}
		
	\section{Modulação em baixo nível (C/C++)}
	\paragraph{}
		A modulação é realizada através de duas Look Up Tables (LUT) contendo valores de duas funções seno (cada um com uma frequência diferente). Resumidamente: a informação (convertida em binário) é transformada num sinal modulado por FSK.
	\subsection{lut.h}
		\begin{minted}[frame=lines,
		framesep=2mm,
		baselinestretch=1.2,
		bgcolor=bg,
		fontsize=\footnotesize,
		breaklines
		]{c}
		#ifndef LUT_H
		#define LUT_H
		
		#include <stdlib.h>
		#include <stdio.h>
		
		#define DATA_SIZE 10
		
		#define VECTOR_SIZE 6400
		
		static const int lut_2T[] = {
		0x8,0x9,0xa,0xa,0xb,0xc,0xc,0xd,
		0xe,0xe,0xf,0xf,0xf,0x10,0x10,0x10,
		0x10,0x10,0x10,0x10,0xf,0xf,0xf,0xe,
		0xe,0xd,0xc,0xc,0xb,0xa,0xa,0x9,
		0x8,0x7,0x6,0x6,0x5,0x4,0x4,0x3,
		0x2,0x2,0x1,0x1,0x1,0x0,0x0,0x0,
		0x0,0x0,0x0,0x0,0x1,0x1,0x1,0x2,
		0x2,0x3,0x4,0x4,0x5,0x6,0x6,0x7};
		
		static const int lut_T[] = {
		0x8,0xa,0xb,0xc,0xe,0xf,0xf,0x10,
		0x10,0x10,0xf,0xf,0xe,0xc,0xb,0xa,
		0x8,0x6,0x5,0x4,0x2,0x1,0x1,0x0,
		0x0,0x0,0x1,0x1,0x2,0x4,0x5,0x6};
		
		void lut_association(int input_binary_signal[], int **output_analog_signal);
		
		#endif
		\end{minted}
		
	\subsection{lut.c}
		\begin{minted}[frame=lines,
		framesep=2mm,
		baselinestretch=1.2,
		bgcolor=bg,
		fontsize=\footnotesize,
		breaklines
		]{c}
		#include "lut.h"
		
		void lut_association(int input_binary_signal[], int **output_analog_signal)
		{
			free(*output_analog_signal);
			*output_analog_signal = malloc(VECTOR_SIZE * sizeof(int));
			if (*output_analog_signal == NULL)	return;
			for (int i = 0; i < DATA_SIZE; i++)	{
				int multiple1 = i * DATA_SIZE * 64;
				if (input_binary_signal[i] == 1)	{
					for (int j = 0; j < DATA_SIZE; j++)	{
						int multiple2 = j * 64;
						for (int k = 0; k < 64; k++)	{
							
							(*output_analog_signal) [multiple1 + multiple2 + k] = lut_T[k % 32];
						}
					}
				}	else	{
					for (int j = 0; j < DATA_SIZE; j++)	{
						int multiple2 = j * 64;
						for (int k = 0; k < 64; k++)	{
							(*output_analog_signal) [multiple1 + multiple2 + k] = lut_2T[k];
						}
					}
				}
			}
		}
		\end{minted}
		
	\subsection{mod\_main.c}
		\begin{minted}[frame=lines,
		framesep=2mm,
		baselinestretch=1.2,
		bgcolor=bg,
		fontsize=\footnotesize,
		breaklines
		]{c}
		#include "lut.h"
		
		int main(void)
		{
			int transmitted_data[DATA_SIZE] = {1, 0, 1, 0, 1, 1, 1, 0, 0, 1};
			int *modulated_signal;
		
			modulated_signal = NULL;
			lut_association(transmitted_data, &modulated_signal);
			free(modulated_signal);
		
			return 0;
		}
		
		\end{minted}
		
	\section{Aquisição dos dados}
	
		\begin{minted}[frame=lines,
		framesep=2mm,
		baselinestretch=1.2,
		bgcolor=bg,
		fontsize=\footnotesize,
		breaklines
		]{python}
		import time
		import board
		import busio
		import adafruit_ads1x15.ads1115 as ADS
		from adafruit_ads1x15.ads1115 import Mode
		from adafruit_ads1x15.analog_in import AnalogIn
		
		RATE = 860
		SAMPLES = 10000
		
		# BUS I2C de alta frequencia
		i2c = busio.I2C(board.SCL, board.SDA, frequency=1000000)
		ads = ADS.ADS1115(i2c)
		
		#Pino A0
		channel0 = AnalogIn(ads, ADS.P0)
		
		ads.mode = Mode.CONTINUOUS
		ads.data_rate = RATE
		ads.gain = 2/3
		
		
		adc_values = [None]*SAMPLES
		
		start = time.monotonic()
		
		# Lê
		for i in range(SAMPLES):
		adc_values[i] = channel0.value
		#adc_values[i] = channel0.voltage
		end = time.monotonic()
		total_time = end - start
		
		with open('values.txt', 'w') as f:
		for item in adc_values:
		f.write("%s " % item)
		
		print("Tempo de captura: {}s".format(total_time))
		\end{minted}


	\section{Demodulação em baixo nível (C/C++)}
	\paragraph{}
		A demodulação consiste em, a partir dos dados obtidos pelo sensor, interpretar o sinal recebido e identificar a frequência com que esse "cruza" a faixa de DC OFFSET. Calcula-se o valor médio da tensão gerada pelo sinal e, a partir desse, consegue-se identificar diferentes frequências.
	
	\subsection{functions.h}
		\begin{minted}[frame=lines,
		framesep=2mm,
		baselinestretch=1.2,
		bgcolor=bg,
		fontsize=\footnotesize,
		breaklines
		]{c}
		#include <stdio.h>
		#include <stdlib.h>
		
		#define BITS 10
		#define PERIOD_SAMPLE 2560
		#define LENGTH (BITS*PERIOD_SAMPLE)
		
		void get_wave(int array[], int size);
		
		void analyze_zeros(int wave[], int **zeros, const int nro_bits, const int sampling_period_length, double reference);
		
		double get_mean(int vector[], int vector_size);
		
		void get_normalized(double normalized_vector[], int vector[], double mean);
		
		void print_vector(double vector[]);
		\end{minted}
		
	
	\subsection{functions.c}
		\begin{minted}[frame=lines,
		framesep=2mm,
		baselinestretch=1.2,
		bgcolor=bg,
		fontsize=\footnotesize,
		breaklines
		]{c}
		#include "functions.h"
		
		void get_wave(int array[], int size)
		{
		FILE *myfile;
		
		myfile=fopen("new_mod_result.txt", "r");
		
		for(int i = 0; i < LENGTH; i++)   {
		fscanf(myfile,"%d", &array[i]);
		}
		
		fclose(myfile);
		
		}
		
		void analyze_zeros(int wave[], int **zeros, const int nro_bits, const int sampling_period_length, double reference)
		{
			free(*zeros);
			*zeros = malloc(sizeof(int) * nro_bits);
			if (*zeros == NULL) return;
			int zero_sampling[nro_bits];
			int positive = 0;
			int negative = 0;
			int dif_signal = 0;
			
			for (int i = 0; i < nro_bits; i++)  {
				int vector_position = PERIOD_SAMPLE * i;
				zero_sampling[i] = 0;
				for (int j = 0; j < sampling_period_length; j++) {
					if (dif_signal > nro_bits)  {
						if (wave[vector_position + j] > reference)   {
							positive = 1;
						}
						if (wave[vector_position + j] < reference)   {
							negative = 1;
						}
						if (positive && negative)   {
							zero_sampling[i] += 1;
							positive = 0;
							negative = 0;
							dif_signal = 0;
						}
					}
					dif_signal++;
				}
			}
			for (int i = 0; i < BITS; i++){
				(*zeros)[i] = zero_sampling[i];
			}
		}
		
		double get_mean(int vector[], int vector_size)
		{
			double sum = 0;
			for (int i = 0; i < vector_size; i++)   {
				sum += vector[i];
			}
			
			return sum/vector_size;
		}
		
		void get_normalized(double normalized_vector[], int vector[], double mean)
		{
			for (int i = 0; i < BITS; i++)   {
				normalized_vector[i] = (double)vector[i] / mean;
			}
		}
		
		void print_vector(double vector[])
		{
			printf("Demodulated vector: [ ");
			for (int i = 0; i < BITS; i++)    {
				if (vector[i] > 1)    {
					printf("1 ");
				}   else    printf("0 "); 
			}
			printf("]\n");
		}
		\end{minted}
		
		
	\subsection{demod\_main.c}
		\begin{minted}[frame=lines,
		framesep=2mm,
		baselinestretch=1.2,
		bgcolor=bg,
		fontsize=\footnotesize,
		breaklines
		]{c}
		#include "functions.h"
		
		int main(void)
		{
			printf("Original vector (obtained from modulation in C): \
			[ 1 0 1 0 1 1 1 0 0 1 ]\n");
			
			int array[LENGTH];
			int *zeros;
			double normalized_vector[BITS];
			int demodulated_wave[BITS];
			
			get_wave(array, LENGTH);
			
			zeros = NULL;
			analyze_zeros(array, &zeros, BITS, PERIOD_SAMPLE, get_mean(array, LENGTH));
			
			double mean = get_mean(zeros, BITS);
			get_normalized(normalized_vector, zeros, mean);
			print_vector(normalized_vector);
			
			free(zeros);
			
			return 0;
		}
		\end{minted}
		
	

	
		
\end{document}